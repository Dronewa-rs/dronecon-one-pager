\documentclass[10pt, letterpaper]{article}

\input{cv-themes/theme_selection.tex}

%----------------------------------------------------------------------------------------
%	 THEMES
%----------------------------------------------------------------------------------------

% Define the desired theme out of the following: beige, blue, bw, coral, earth, framed, gray, minimal, onyx, plain
% See screenshots in preview/ directory
\theme{beige}

%----------------------------------------------------------------------------------------
%	 PERSONAL INFORMATION
%----------------------------------------------------------------------------------------

% If you don't need a particular field, just remove the content leaving the command, e.g. \aboutdesc{}

\name{EGLIN HACKATHON} % Your name
\jobtitle{Accelerating uncrewed technology} % Job title/career
% \profilepic{dummy_pp.jpg} % Profile picture (supported only with "bw", "gray" and "framed" themes)
\aboutdesc{A hackathon is a 1-2 week event. Teams make new things.  They "hack" together a solution, even when expertise is limited. \newline \newline Dronecon 2024 hopes to achieve a physical, kinetic, and data hackathon. Innovators will gather to accelerate change, modifying civilian and military equipment, testing them in new ways.} % Description for ABOUT ME section

\location{Eglin AFB, FL} % Address/location
\phone{850-883-6347} % Phone number
\mail{96TW.PA.OfficeMailAccount@us.af.mil} % Mail
\dateofbirth{} % Date of birth
\drivinglicenses{} % Drivers license category
\linkedin{} % \linkedin{\href{LINK}{DESCRIPTION}}
\github{} % \github{\href{{LINK}{DESCRIPTION}}

%----------------------------------------------------------------------------------------
%	 SKILLS
%----------------------------------------------------------------------------------------

% Both proskills and perskills can be used separately or together. If you do not plan to use professional and personal skill charts, just remove the content leaving the command, e.g. \perskills{} or \proskills{}

% Define professional skills (values are from interval [0,1])
\proskills{}
%\proskills{}

% Define personal skills (values are from interval [0,1])
\perskills{}
%\perskills{}

% Define language skills (values are from interval [0,5). If you do not plan to use the language skills chart, just remove the content leaving the command, e.g. \langskills{}
% Language skill circles are designed to be included into the sidebar
\langskills{{AI/4},{Analytics/4},{Modeling/3},{Robotics/4},{Aviation/5},{Land/3},{Sea/4},{Cyber/2},{Spectrum/3},{Testing/5}}
%\langskills{}

\begin{document}

\makeprofile % Print name & job description. Also prints out profile picture if it's supported by the theme

\begincols

%----------------------------------------------------------------------------------------
%	 SIDEBAR
%----------------------------------------------------------------------------------------
% Use \subsection inside the sidebar

% Print defined contact information
% \makecontact{NAME FOR CONTACT SECTION}
\makecontact{CONTACT INFO}

% \makedob{NAME FOR DATE OF BIRTH SECTION}
\makedob{DATE OF BIRTH}

% \makelicense{NAME FOR ABOUT ME SECTION}
\makeabout{WHAT IS IT?}

% \makelicense{NAME FOR DRIVING LICENSE SECTION}
\makelicense{DRIVING LICENSE}

\subsection{FEATURES}
\langcircles % Command for drawing language skill circles

% \subsection{PERSONAL}
% \perskills % Command for drawing personal skill bars.

%\subsection{PROFESSIONAL}
%\proskills % Command for drawing professional skill bars.

%----------------------------------------------------------------------------------------
%	 BODY
%----------------------------------------------------------------------------------------
% Use \section inside the body
\switchcols % This command is used to switch to the document body

\section{GOVERNMENT ACTIVITIES}

% \begin{cvitem}{Education level}{College/High school}{Location}{Duration}
% Description
% \end{cvitem}

\begin{enumerate}

\item Autonomy: Building and automating new paths, methods.

\item Swarming: Offensive and defensive tactics.

\item Cross-agency: Collaborative testing from across DoD and partner agencies.

\item Loitering: New first-person views in combat-like scenarios.

\item Air-ground coordination: Drone-augmented fire team vs. team. 

\item Anti-aircraft: Ground-air shoot-down testing.

\item Blind ops: Testing zero-camera navigation through other sensors.

\item Artificial intelligence: Modeling the automation schools of thought, neural networks, machine learning, deep learning, testing one against the other.  Modeling and testing how they might work together in teams. 

\item Mapping: Translating data from drone to actionable data. 

\item Electromagnetic signals: Weaponized, deployed from each team.

\item Network security: Network sniffing, offensive cyber-from-drone attacks. 

\item Disaster coordination: First responders, package delivery, area monitoring.

\end{enumerate}

\section{ACADEMIA, INDUSTRY}

\begin{enumerate}
	
	\item Data analytics: Forecasting drone behavior, swarms. Accepting unclassified data from DoD end-users to find new insights and efficiencies.
	
	\item Rapid prototyping: Creating new theoretical solutions to drone, battlefield problems in the modern era. Opportunity for unfettered creativity.
	
	\item Theory-testing: "Hack" together drone solutions based on said prototypes.
	
	\item Education partnership: Crafting DIY drones using single-board computers, radios. Motorized folded paper planes. Scratch origami paper for participants to make folded paper creations while boosting their creativity.
	
	\item Volunteer aircraft: Testing industry equipment in hazardous scenarios. 
	
	\item Psyops: Influence campaigns between teams, leaflet dropping, skywriting. 
	
	\item Social science: Innovation research, optimizing creativity, personality mixes. Participants opting-in to wearing wrist-bands to track attendee behavior (agent modeling), safety trends and risk acceptance tracking, crowd theory (individual decision-making), and rule-following (normalization of deviance).

\end{enumerate}

\end{document}